\documentclass[]{article}
\usepackage{lmodern}
\usepackage{amssymb,amsmath}
\usepackage{ifxetex,ifluatex}
\usepackage{fixltx2e} % provides \textsubscript
\ifnum 0\ifxetex 1\fi\ifluatex 1\fi=0 % if pdftex
  \usepackage[T1]{fontenc}
  \usepackage[utf8]{inputenc}
\else % if luatex or xelatex
  \ifxetex
    \usepackage{mathspec}
  \else
    \usepackage{fontspec}
  \fi
  \defaultfontfeatures{Ligatures=TeX,Scale=MatchLowercase}
\fi
% use upquote if available, for straight quotes in verbatim environments
\IfFileExists{upquote.sty}{\usepackage{upquote}}{}
% use microtype if available
\IfFileExists{microtype.sty}{%
\usepackage{microtype}
\UseMicrotypeSet[protrusion]{basicmath} % disable protrusion for tt fonts
}{}
\usepackage[margin=1in]{geometry}
\usepackage{hyperref}
\hypersetup{unicode=true,
            pdftitle={Max Kovalovs},
            pdfborder={0 0 0},
            breaklinks=true}
\urlstyle{same}  % don't use monospace font for urls
\usepackage{graphicx,grffile}
\makeatletter
\def\maxwidth{\ifdim\Gin@nat@width>\linewidth\linewidth\else\Gin@nat@width\fi}
\def\maxheight{\ifdim\Gin@nat@height>\textheight\textheight\else\Gin@nat@height\fi}
\makeatother
% Scale images if necessary, so that they will not overflow the page
% margins by default, and it is still possible to overwrite the defaults
% using explicit options in \includegraphics[width, height, ...]{}
\setkeys{Gin}{width=\maxwidth,height=\maxheight,keepaspectratio}
\IfFileExists{parskip.sty}{%
\usepackage{parskip}
}{% else
\setlength{\parindent}{0pt}
\setlength{\parskip}{6pt plus 2pt minus 1pt}
}
\setlength{\emergencystretch}{3em}  % prevent overfull lines
\providecommand{\tightlist}{%
  \setlength{\itemsep}{0pt}\setlength{\parskip}{0pt}}
\setcounter{secnumdepth}{0}
% Redefines (sub)paragraphs to behave more like sections
\ifx\paragraph\undefined\else
\let\oldparagraph\paragraph
\renewcommand{\paragraph}[1]{\oldparagraph{#1}\mbox{}}
\fi
\ifx\subparagraph\undefined\else
\let\oldsubparagraph\subparagraph
\renewcommand{\subparagraph}[1]{\oldsubparagraph{#1}\mbox{}}
\fi

%%% Use protect on footnotes to avoid problems with footnotes in titles
\let\rmarkdownfootnote\footnote%
\def\footnote{\protect\rmarkdownfootnote}

%%% Change title format to be more compact
\usepackage{titling}

% Create subtitle command for use in maketitle
\newcommand{\subtitle}[1]{
  \posttitle{
    \begin{center}\large#1\end{center}
    }
}

\setlength{\droptitle}{-2em}

  \title{Max Kovalovs}
    \pretitle{\vspace{\droptitle}\centering\huge}
  \posttitle{\par}
    \author{}
    \preauthor{}\postauthor{}
    \date{}
    \predate{}\postdate{}
  
\usepackage{enumitem} \usepackage{multicol} \usepackage{graphicx}
\usepackage{lipsum} \usepackage{array, xcolor} \usepackage{geometry}
\definecolor{lightgray}{gray}{0.8}
\newcolumntype{L}{>{\raggedleft}p{0.17\textwidth}}
\newcolumntype{R}{p{0.8\textwidth}}

\newcommand\VRule{\color{lightgray}\vrule width 0.5pt}

\pagenumbering{gobble}

\subtitle{max@maxkovalovs.us $\cdot$ github.com/max-kov}

\geometry{bottom=1cm,top=1cm}

\begin{document}
\maketitle

\section*{Employement}
\begin{tabular}{L!{\color{lightgray}\vrule width 0.5pt}R}
Summer 2018&{\bf TNG Technology Consulting} \hfill \textit{Software Engineer Intern}\\
&Developed a REST API with Flask for working with a large set of legal documents.
\begin{itemize}
\item Created a continuous integration pipeline with Jenkins, tox and pytest.
\item Designed a normalised multi-table database schema to store document information.
\item Dockerised the application for easy deployment.
\item Developed a front end for uploading documents into the system.
\item Performed data analysis of the legal document dataset using Pandas.
\item Acquired experience working in agile teams, writing good tests, presenting and talking to clients.
\end{itemize}
\end{tabular}

\section*{Education}
\begin{tabular}{L!{\color{lightgray}\vrule width 0.5pt}R}
2018 - present&{\bf Imperial College London} \hfill \textit{BEng. Computing}\\
2016 - 2018&{\bf Chilwell Sixth Form} \hfill \textit{A-Levels}\\
&Mathematics (A*), physics (A*), computing (A) and further mathematics (A) A-Levels. Participated in hackathons and Olympiads.
\end{tabular}

\section*{Projects}
\begin{tabular}{L!{\color{lightgray}\vrule width 0.5pt}R}
2017&{\bf Pool Game}\hfill \textit{github.com/max-kov/pool}\\
&A 2d top-down pool game written in python.
\begin{itemize}
\item Realistic ball collisions based on Newtonian mechanics.
\item Code analysis and testing pipelines with CodeClimate and TravisCI.
\item Custom 3d ball rendering.
\end{itemize}\\
2018&{\bf The Archived}\hfill \textit{maxkovalovs.us}\\
&Personal blog on programming-related topics.
\begin{itemize}
\item Written in RMarkdown (Blogdown).
\item Seamless compilation and deployment with TravisCI.
\item Hosted on GitHubPages.
\end{itemize}
\end{tabular}

\section*{Extra-curricular}
\begin{tabular}{L!{\color{lightgray}\vrule width 0.5pt}R}
2013 - 2016&{\bf Progmeistars}\hfill \textit{Computer science school}\\
&\begin{itemize}
\item Took several programming courses at Progmeistars, where I learned the basics of programming. I did courses on graphics, OOP, algorithms and others.
\item Was a part of a junior IOI team. We participated in many solo and team events. To prepare for olympiads we had weekly lessons where we've covered computer science topics like big O notation, sorting algorithms and trees.
\end{itemize}\\
July 2017&{\bf Nottingham Trent University}\hfill \textit{Experience}\\
&Spent a week with PhD students developing a VR model of Nottingham. The model was made in unity engine and I was working on vehicle movement.
\end{tabular}


\end{document}
